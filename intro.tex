
\section{Introduction}
\label{sec:intro}

The Fourth Industrial Revolution is characterized by technological advancements in high-speed Internet, artificial intelligence, big data analytics, and cloud computing~\citep{schwab_fourth_2017}. Along with its emergence and progress, numerous digital artifacts (e.g., sensors, tools, and information systems) capable of producing or collecting data are implemented by organizations worldwide to modernize their services, scale their business, or improve the efficiency of data exchange. Meanwhile, these digital artifacts record massive amounts of data describing the context and outcomes of users' actions and forming each user's unique digital traces~\red{not found Hedman, 2013}. As part of this trend, the digital traces of user-generated content on social media represent a massive and novel source of ecological data that describes human behaviors and psychological characteristics~\citep{foster_managing_2016, settanni_predicting_2018} which can be used for developing novel disease surveillance systems~\citep{wenli_zhang_comprehensive_2020}. 


In the 1st century, depression is one of the major contributors to the overall global disease burden\footnote{A concept developed by the Harvard school of public health, WHO, and the World Bank, which is used to calculate disability, premature death, and other factors.}. An estimated 3.8 percent of the world's population suffers from depression~\citep{who_who_2022}. In the United States (US), more than 21 million adults have had depression, representing 8.4\% of all US adults~\citep{nih_major_2022}, and approximately \$225 billion is spent on depression-related treatments and services every year~\citep{openminds_us_2019}. Meanwhile, depression is also the primary cause of lost productivity, with employers in the US losing more than \$23 billion each year as a result of its effects on absenteeism and presenteeism~\citep{nadeem_identifying_2016}\footnote{When employees are present for work but less productive due to their illness.}


Although considerable effort has been devoted to this research area, depression is still difficult to diagnose~\citep{andrade_epidemiology_2003}. First, there is no reliable laboratory test for diagnosing depression; the diagnoses are normally based on patients' self-reported experiences. Second, current electronic health record (EHR) systems lack tracking of behavioral data for effective depression detection. Third, people with depression have depressive and non-depressive episodes which further complicate accurate diagnosis. Meanwhile, depression continues to be underdiagnosed~\citep{prince_no_2007}. First, myths and misunderstandings cause individuals and primary care physicians to be unaware of depression symptoms. Second, the stigma associated with depression causes people to hide the issues and delay help-seeking. Third, people suffering from depressive disorders may be unable to access health care services due to a lack of transportation, financial resources, or insurance.

Improving approaches for detecting and surveilling depression is key to combating depression and has far-reaching health and societal implications. The rapid proliferation of social media and the associated social media users' digital traces open up new possibilities in depression detection. A diverse set of quantifiable signals relevant to depression, including self-reported diagnosis, causes, and symptoms, are observed on social media~\citep{coppersmith_measuring_2014,coppersmith_quantifying_2014, nadeem_identifying_2016}. Research suggests that health digital traces on social media provide a great means for capturing an individual's current states of mind, feelings, behaviors, and activities that often characterize depression~\citep{choudhury_predicting_2013,nadeem_identifying_2016}. Existing research shows that social media-based depression screening has the potential to achieve prediction results that are comparable to unaided clinician assessment and screening surveys~\citep{guntuku_detecting_2017}. Meanwhile, unlike traditional survey- or interview-based depression screening approaches, which take a snapshot of a patient's mental health conditions, social media-based techniques continuously collect digital traces that allow the tracking of an individual's mental health over time. Meanwhile, digital trace data enables non-reactive research, in which data is not generated through contact between researchers and subjects~\citep{salganik_bit_2019}, logging ``naturally" occurring evidence of social behavior or psychological traits. 

Using social media users' digital trace data, social media-based depression screening can facilitate novel approaches to fight depression and eventually relieve its social and economic burden. For social media users, such methods can provide early detection and raise awareness for those who are at risk of clinical depression. For social media platforms, depression screening techniques enable them to develop new services with personalized recommendations for users with depression (e.g., encourage the identified users to seek help and treatments; promote educational content and tools, treatment options, social support, etc.)\red{For public administration, xxx }

While promising, the current mainstream approaches for detecting depression on social media have major limitations. Most of these methods use feature engineering for depression detection. However, the adopted features, including LIWC, n-grams, and sentiment analysis~\citep{gautier_environmental_2017}\red{M's?}, do not specify clinical depression assessment, therefore deteriorating depression prediction performance. With the development of deep learning, an increasing number of researchers turn to end-to-end sequence models for depression detection using raw social media posts~\citep{lin_first_2020}\red{baseline table}. However, limitation persists due to depressive disorders largely displaying subtle and implicit changes in language and behavior, such as a switch in the types of topics~\citep{coppersmith_measuring_2014,coppersmith_quantifying_2014}. Such patterns are difficult to model unless massive labeled data is available for the training of deep learning models. A small group of researchers relies on social media users' online behaviors: e.g., the engagement of social media usage, the changes in egocentric networks~\red{Choudhury et al., 2013}, dictionary-based symptom and drug taggers~\citep{hussain_exploring_2020}. However, these predefined features do not align with the medical knowledge in depression diagnosis and do not significantly improve prediction performance. Moreover, the symptomatic episodes of depression come and go. Patients do not exhibit depressive symptoms consistently on social media. All the existing methods lack effective means of capturing the dynamic patterns that characterize depression.

To cope with the shortcomings of the previous approaches, guided by the medical knowledge in depression diagnosis, we devise a named entity recognition (NER) and knowledge-aware LSTM depression detection method using digital trace data on social media. Our work has important implications for information systems (IS) research. First, positioned in the computational design science research, we propose a knowledge-driven machine learning method that sheds light on other domain-knowledge-rich areas in IS, such as healthcare, finance, and legal research, in which domain knowledge can be meaningfully incorporated into machine learning to develop novel and impactful artifacts~\citep{abbasi_towards_2020}. The proposed method is our major contribution to the IS knowledge base~\citep{padmanabhan_machine_2022}. Second, the IT artifact of interest in this study is digital traces on social media, and we demonstrate that digital traces are valuable data sources for decision-making, particularly in healthcare analytics. Third, our work expands NLP research in IS, which is situated at the crossroads of design and data science and represents significant opportunities to develop effective artifacts that leverage unstructured digital trace data to address socio-technical concerns~\citep{yang_getting_2022,abbasi_big_2016}. Meanwhile, our work has significant practical implications for depression detection and surveillance. It provides an accurate prediction method that can provide complementary information to existing clinical depression screening procedures. From the perspective of public health and disaster management, our method can enable large-scale analyses of a population's mental health conditions beyond what has previously been possible with traditional methods.